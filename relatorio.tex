\documentclass[a4paper,10pt]{article}
\usepackage[brazilian]{babel}
\usepackage[utf8]{inputenc}
\usepackage[T1]{fontenc}
\usepackage{amsmath}
\usepackage{amssymb}

\title{Sistemas de recomendações}
\author{Alex Grilo \\ Orientador Flávio Keidi Miyazawa\\ MC032 - Estudo Dirigido \\ \normalsize{Instituto de Computação -- Universidade Estadual de Campinas}}

\begin{document}

\maketitle

\newtheorem{definicao}{Definição}
\newtheorem{lema}{Lema}
\newtheorem{coro}{Corolário}


\section{Introdução}

Um sistema de recomendação considera quais produtos cada usuário
escolheu no passado e tenta deduzir que outros produtos um determinado
usuário pode estar interessado. 

O principal interesse deste trabalho é investigar este problema sob a
abordagem algorítmica.

\section{Competitive Recommendation Systems}


\subsection{Introdução}
A ideia basica da primeira abordagem é baseada nas técnicas de 
reconstru\c{c}\~ao da matriz a partir de informa\c{c}\~oes parciais
da mesma. 

Para isso, utiliza-se a técnica de SVD e 

\subsection{Notação}

$A$ : matriz de recomendação original
$A_(i)$ : $i-ésima$ linha da matriz A. No caso da matriz de recomendações, é o vetor de utilidades do $i-ésimo$ usuário.
$A_{ij}$ : valor da utilidade do projeto j para o usuário i
$a_{ir}$ : valor da utilidade do $r-ésimo$ produto com maior utilidade para o usuário i 

\subsection{Definições}


\begin{definicao} \label{definicao:box} Um produto j é dito bom se para um usuário $i$ se $A_{ij} > a_{ir} - \delta$para o $r$ constante e $\delta$ pequeno.
\end{definicao}
\begin{definicao} \label{definicao:box}Uma recomendação é dita boa para um usuário $i$ se contém pelo menos um produto bom para aquele usuário.
\end{definicao}

\subsection{Reconstrução de matrizes e boas recomendações}

\begin{lema} \label{lema:box}
Dado que existe uma aproximação  tal que $\Vert A - $Â$ \Vert \le \epsilon \Vert A \Vert^2_F$ a probabilidade de uma recomenda\c{c}\~ao
ruim é
 
$Pr [ $recomendação ruim$ ] \leq \frac{2\varepsilon}{r\delta^2}$

\end{lema}

\paragraph{Ideia da prova:}
  
A menor contribuição que uma má recomendação contribui para o erro de $\Vert A - \^A \Vert \le \Vert A \Vert^2_F$ é quando os $r$ produtos com utilidade maior de $A_{(i)}$ possuem utilidade $u$ e os $r$ seguinte produtos com utilidade maior possuem utilidade $u - \delta$.
 
   Em $Â_{(i)}$, para a recomendação ser ruim, os $r$ produtos mais avaliados possuem utilidade $\leq x$ e os $r$ produtos seguintes possuem utilidade $\geq x$. Isso acontece quando $x = u - \frac{\delta}{2}$. 

   Calculando o erro a partir desses dados, assumindo que $\lambda m$ usuários tem uma recomendação ruim e sabendo que o erro é limitado por $\epsilon \vert A \vert^2_F$, obtem-se o resultado proposto no enunciado.

\subsection{Modelo de usuários}

Vamos assumir a existência de $l$ tipos de usuários, caracterizados
pelos vetores $v^{(1)}$, ..., $v^{(l)} \in $ [$0,1$]$^n$
, onde cada vetor possui
tamanho 1 e são "bem separados". Intuitivamente, isso significa que os tipos
são linearmente independentes. 

\begin{definicao} \label{definicao:box} Os vetores $v^{(1)}$, ..., $v^{(l)}$ são chamados $\delta $ -separados se para
cada par $(i,j)$ tal que $i \neq j$, $v^{(i)} \dot v^{(j)} \leq \delta $
\end{definicao}

Chamaremos $t_j$ o número de usuários do tipo $j$. 

Assumiremos que os tipos estão ordenados pela ordem decrescente do número de usuários e que os usuários estão ordenados na matriz $A$ na ordem crescente do tipo ao qual pertencem.  

\begin{definicao} \label{definicao:box} Uma matriz de preferência A é dita $(\lambda,k)$-efetiva se 

$\sum^k_{i=1} t_i \geq \lambda m$
\end{definicao}

\begin{lema} \label{lema:box}
Para uma matriz $(\lambda,k)$-efetiva na forma de A  $\Vert A - $Â$ \Vert \le ( 1 - \lambda ) \Vert A \Vert^2_F$ 
\end{lema}
\paragraph{Ideia de prova:}

Dado que $A_k$ é a melhor aproximação de posto $k$ de A, pela construção acima da matriz $A$, a matriz $B_k$ consistente somente dos $k$ primeiros tipos da matriz possui aproximação 
$\Vert A - B_k\Vert^2_F \leq ( 1 - \lambda ) \Vert A \Vert^2_F$.

Portanto a matriz $A_k$ tem uma aproximação pelo menos igual a anterior.

\subsection{Desvio nas utilidades dos produtos}


\begin{definicao} \label{definicao:box} Os vetores $v^{(1)}$, ..., $v^{(l)}$ são chamados $\delta $ -separados se para
cada par $(i,j)$ tal que $i \neq j$, $v^{(i)} \dot v^{(j)} \leq \delta $
\end{definicao}
Consideraremos que a partir da matriz $A$ proposta na subseção anterior, é adicionado um erro, pois os produtos não possuem exatamente a mesma utilidade para todos usuários do mesmo tipo. Modelaremos o erro adicionando à matriz $A$ uma matriz de erro $E$ tal que $E_ij$ é uma variável aleatória de média 0 e variância $O(\frac{\epsilon^2}{m} + n )$ para $0 < \epsilon < 1$.

Chamaremos de à a matriz $ A + E $.

\subsection{Limitantes na aproximação de posto menor}

O primeiro objetivo é provar que Ã$_k$ e $A$ são próximas, assim poderemos recriar de forma eficiente a matriz $A$ e fazer boas recomendações de produtos. 

\begin{lema}
 Se $\sigma_1, ..., \sigma_k$ os valores singulares de $A$, $A$ é $\delta$-separada com $\delta = O(\frac{1}{n})$ e $\frac{t_k}{t_{k+1}} \geq \beta_1 \frac{t_1}{t_k}$, onde $\beta_1$ é uma constante grande, então 

$\frac{\sigma_k}{\sigma_{k+1}} \geq \beta_2 \frac{\sigma_1}{\sigma_{k}}$
para alguma constante $\beta_2 = O (sqrt(\beta_1))$.

\end{lema}
\paragraph{Ideia da prova:} Analizando $A_\delta$ e $A_0$, pode-se verificar que $A_\delta A_\delta^T - A_0 A_0^T$ possui 2-norma no máximo $n \delta$.

Aplicando teoria padrão de perturbação para valores singulares de matrizes simétricas, pode-se concluir que os valores singulares de $A_\delta$ são perturbados somente por uma constante. Escolhendo $\beta_1$ e $\beta_2$ cuidadosamente o resultado é válido. 


Utilizando o lema acima e os lemas do TODO:por bibliografia, obtemos os seguintes corolários:
  
\begin{coro} 
Como $\vert E \vert_2 = O(\epsilon)$, com probabilidade  $ 1 - o(1) $

$\vert ($Ã$_k)_{(i)} - (A_k)_{(i)}\vert_2 \geq O(\epsilon)$ e \end{coro}

\begin{coro} Com probabilidade $1- o(1)$, temos 

$\Vert$ Ã$_k - A_k\Vert^2_F \geq O(\epsilon)m \geq O(\epsilon)\Vert A \Vert^2_F $
\end{coro}

E se a matriz A é $(\lambda, k)$-efetiva, segue que 
\begin{coro}

$\Vert$ Ã$_k - A_k\Vert^2_F \geq  O(\epsilon + 1 - \lambda)\Vert A \Vert^2_F $
\end{coro}

\section{Ã desconhecida}

Apesar de termos encontrado que Ã$_k$ e $A$ são próximas, um problema é que a matriz à é desconhecida. Nós buscamos um algoritmo que com amostra de $O(m+n)$ elementos, se consiga reconstruir a matriz e dar boas recomendações. 

Assume-se que cada tipo efetivo 

\addcontentsline{toc}{section}{\bibname}
\bibliographystyle{plain}
\bibliography{balancedAllocation}
\end{document}
