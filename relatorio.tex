\documentclass[a4paper,10pt]{article}
\usepackage[brazilian]{babel}
\usepackage[utf8]{inputenc}
\usepackage[T1]{fontenc}
\usepackage{amsmath}
\usepackage{amssymb}
\usepackage{amsthm}
\usepackage{hyperref}
\usepackage{graphicx}

\hypersetup{pdftitle={Sistemas de recomendações}}
\hypersetup{pdfauthor={Alex Grilo}}

\title{Sistemas de recomendações}
\author{Alex Grilo \\ Orientador Flávio Keidi Miyazawa\\ MC032 - Estudo Dirigido \\ \normalsize{Instituto de Computação -- Universidade Estadual de Campinas}}

\numberwithin{equation}{section} %sets equation numbers <chapter>.<section>.<index>

\begin{document}

\maketitle

\section{Introdução}

Um sistema de recomendação considera quais produtos cada usuário
escolheu no passado e tenta deduzir que outros produtos um determinado
usuário pode estar interessado. 

O principal interesse deste trabalho é investigar este problema sob a
abordagem algorítmica.

\section{}

\phantomsection
\addcontentsline{toc}{section}{\bibname}
\bibliographystyle{plain}
\bibliography{balancedAllocation}
\end{document}
