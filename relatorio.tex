\documentclass[a4paper,10pt]{article}
\usepackage[brazilian]{babel}
\usepackage[utf8]{inputenc}
\usepackage[T1]{fontenc}
\usepackage{amsmath}
\usepackage{algorithmic}
\usepackage{amssymb}

\title{Sistemas de recomendações}
\author{Alex Grilo \\ Orientador Flávio Keidi Miyazawa\\ MC032 - Estudo Dirigido \\ \normalsize{Instituto de Computação -- Universidade Estadual de Campinas}}

\begin{document}

\maketitle

\newtheorem{definicao}{Definição}
\newtheorem{lema}{Lema}
\newtheorem{coro}{Corolário}
\newtheorem{teo}{Teorema}


\section{Introdução}

O objetivo deste relatório é descrever os assuntos estudados durante a 
disciplina de MC032 no segundo semestre de 2010. 

O principal interesse deste trabalho foi pesquisar na listeratura este
problema diferentes abordagens para resolvê-lo.

TODO:explicar estrutura do relatório

\newpage

\section{Sistemas de recomendações}

Um sistema de recomendação considera quais produtos cada usuário
escolheu no passado e tenta deduzir que outros produtos um determinado
usuário pode estar interessado. 

\newpage

\section{Competitive Recommendation Systems}

\subsection{Introdução}

A ideia básica desta abordagem é reduzir o problema de sistema de recomendações
no problema de reconstrução de matrizes a partir de informações parciais da mesma. 
Foram utilizadas estratégias de reconstrução a partir de técnicas de SVD, encontrando
aproximações para a matriz original 

\subsubsection{Divisão em tipos}
É considerado que os usuários podem ser divididos em grupos com interesses semelhantes,
classificando-os em tipos. Consideraremos que $k$ seja o número de tipos, onde $k$ é
uma constante que não depende do número de usuários.

\subsubsection{Qualidade dos algoritmos}

Um bom algoritmo de recomendações deve dar boas recomendações utilizando o mínimo de 
informação possível. Para isso são estabelecidas 2 métricas para algoritmos de 
recomendações: um algoritmo é $c-competitivo$ $para$ $amostragem$ se usa somente $ck$
linhas e colunas da matriz para recomendar mais produtos; um algoritmo é $f-competitivo$ $para$ $utilidade$
se faz boas recomendações a $fm$ usuários, onde $m$ é o número de usuários. 

\subsection{Notação}

\begin{itemize}
\item[$A$] matriz de recomendação original
\item[$A_{ij}$] valor da utilidade do produto $j$ para o usuário $i$
\item[$A_{(i)}$] $i$-ésima linha da matriz $A$, que corresponde ao
vetor de utilidades do \\ $i$-ésimo usuário.
\item[$A^{(i)}$] $i$-ésima coluna a matriz $A$, que correspodne à 
a utilidade de um produto para todos os usuários.
\item[$A_k$] Melhor aproximação com posto $k$ de $A$, obtida através da técnica de SVD.
\item[$a_{ir}$] valor da utilidade do $r$-ésimo produto com maior utilidade
para o \\ usuário $i$ 
\end{itemize}

\subsection{Reconstrução de matrizes}

\subsubsection{Boas recomendações}
\begin{definicao} \label{definicao:box} Um produto j é dito bom para um
usuário $i$ se $A_{ij} > a_{ir} - \delta$para o $r$ constante e $\delta$ pequeno.
\end{definicao}
\begin{definicao} \label{definicao:box}Uma recomendação é dita boa para um usuário
$i$ se contém pelo menos um produto bom para aquele usuário.
\end{definicao}

\subsubsection{Reconstrução de matrizes e boas recomendações}

\begin{lema} \label{lema:box}
Dado que existe uma aproximação  tal que $\Vert A - hat{A} \Vert \le \epsilon \Vert A
\Vert^2_F$, a probabilidade de uma recomenda\c{c}\~ao
ruim é 
\begin{equation}
Pr [ recomenda \c{c}c \tilde ao \;\; ruim ] \leq \frac{2\varepsilon}{r\delta^2}
\end{equation} 

\end{lema}

\paragraph{Ideia da prova\\}

A menor contribuição de uma recomendação ruim para o erro de $\Vert A - 
hat{A} \Vert$ é quando os $r$ produtos com maior utilidade 
de $A_{(i)}$ possuem valor igual a $u$ e os $r$ produtos com maior utilidade
dentre os outros produtos possuem utilidade $u - \delta$.
 
Para uma recomendação ser ruim, os $r$ produtos melhor avaliados de $\hat{A}_{(i)}$
possuem utilidade $\leq x$ e os $r$ produtos seguintes possuem utilidade
$\geq x$. No caso extremo, isso acontece quando $x = u - \frac{\delta}{2}$. 

Calculando o erro a partir desses dados e sabendo que o erro é limitado por $\epsilon 
\Vert A \Vert^2_F$, obtem-se o resultado proposto no enunciado.

\subsection{Modelo de usuários}

\subsubsection{Tipos de usuários}
Assumimos que existem $l$ tipos de usuários, caracterizados
pelos vetores $v^{(1)}$, ..., $v^{(l)} \in $ [$0,1$]$^n$, 
e que cada vetor é unitário. 

Assumimos também qeu os vetores são  "bem separados", o que
intuitivamente significa que os tipos são linearmente independentes. 

\begin{definicao} \label{definicao:box} Os vetores $v^{(1)}$, ..., $v^{(l)}$
são chamados $\delta$ -separados se para cada par $(i,j)$ tal que $i \neq j$,
$v^{(i)} . v^{(j)} \leq \delta $
\end{definicao}

Chamaremos $t_j$ o número de usuários do tipo $j$ e assumiremos que os tipos
estão ordenados pela ordem decrescente do número de usuários.

\begin{definicao} \label{definicao:box} Uma matriz de preferência A é dita $(\lambda,k)$-efetiva se 
\begin{equation}
\sum^k_{i=1} t_i \geq \lambda m
\end{equation}
\end{definicao}

$\\ $

\begin{lema} \label{lema:box}
Para uma matriz $(\lambda,k)$-efetiva na forma de A,  
$\\ \\$
\begin{equation}
 \Vert A - {A}_k \Vert \le ( 1 - \lambda ) \Vert A \Vert^2_F 
\end{equation}
\end{lema}
\paragraph{Ideia de prova\\}

Pela construção da matriz $A$, podemos construir a matriz $B_k$ somente
com os vetores dos primeiros $k$ tipos da matriz. Pelas propriedades de 
SVD, pode-se encontrar que a matriz $B_k$ possui aproximação 

\begin{equation}
\Vert A - B_k\Vert^2_F \leq ( 1 - \lambda ) \Vert A \Vert^2_F.
\end{equation}
\\
Portanto a matriz $A_k$, a melhor aproximação de posto $k$ da matriz A, tem uma
aproximação pelo menos igual a anterior.

\subsection{Desvio nas utilidades dos produtos}

Consideraremos que a partir da matriz $A$ proposta na subseção anterior, é
adicionado um erro, pois no mundo real, os produtos não possuem exatamente
a mesma utilidade para todos usuários do mesmo tipo.

 Modelaremos o erro adicionando à matriz $A$ uma matriz de erro $E$ tal
que $E_{ij}$ é uma variável aleatória de média 0 e variância 
$O(\frac{\epsilon^2}{m} + n )$ para $0 < \epsilon < 1$.

Chamaremos de $\tilde A$ a matriz $ A + E $.

\subsection{Limitantes na aproximação de posto menor}

O primeiro objetivo é provar que Ã$_k$ e $A$ são próximas, assim poderemos
recriar de forma eficiente a matriz $A$ e fazer boas recomendações de 
produtos. 

\begin{lema}
 Se $\sigma_1, ..., \sigma_k$ os valores singulares de $A$, $A$ é $\delta$-separada com $\delta = O(\frac{1}{n})$ e $\frac{t_k}{t_{k+1}} \geq \beta_1 \frac{t_1}{t_k}$, onde $\beta_1$ é uma constante grande, então 
\begin{equation}
\frac{\sigma_k}{\sigma_{k+1}} \geq \beta_2 \frac{\sigma_1}{\sigma_{k}}
\end{equation}
para alguma constante $\beta_2 = O (\sqrt{\beta_1}))$.

\end{lema}
\paragraph{Ideia da prova\\}
Analisando $A_\delta$ e $A_0$, pode-se verificar que $A_\delta A_\delta^T - A_0 A_0^T$ possui 2-norma no máximo $n \delta$.

Aplicando teoria padrão de perturbação para valores singulares de matrizes simétricas, pode-se concluir que os valores singulares de $A_\delta$ são perturbados somente por uma constante. Escolhendo $\beta_1$ e $\beta_2$ cuidadosamente o resultado é válido. 

\begin{lema} 
Como $\vert E \vert_2 = O(\epsilon)$, com probabilidade  $ 1 - o(1) $
\begin{equation}
\vert (\tilde A_k)_{(i)} - (A_k)_{(i)}\vert_2 \geq O(\epsilon)
\end{equation}
\end{lema}

\paragraph{Ideia da prova \\}
Utilizando o lema anterior e manipuções algébricas, pode-se
encontrar o resultado proposto


$\\ \\ \\$
Utilizando o lema acima e os lemas do TODO:por bibliografia, obtemos os seguintes corolários:
  
\begin{coro} Com probabilidade $1- o(1)$, temos 
\begin{equation}
\Vert \tilde A_k - A_k\Vert^2_F \geq O(\epsilon)m \geq O(\epsilon)\Vert A \Vert^2_F
\end{equation}
\end{coro}
$\\ $
\begin{coro}
E se a matriz A é $(\lambda, k)$-efetiva, segue que 

\begin{equation}
\Vert \tilde A_k - A_k\Vert^2_F \geq  O(\epsilon + 1 - \lambda)\Vert A \Vert^2_F 
\end{equation}
\end{coro}

\subsection{Ã desconhecida}

Apesar da demonstração que $\tilde A_k$ e $A$ são próximas, um problema é que a
matriz $\tilde A$ é desconhecida. Nós buscamos agora um algoritmo que com amostra de
$O(m+n)$ elementos, consiga reconstruir a matriz e dar boas recomendações.

Assume-se que cada tipo efetivo contenha pelo menos $\frac{\lambda m}{100k}$ usuários.


\subsubsection{Algoritmo}
1. Escolher uniformemente $ak$ usuários e escolher a utilidade de todos produtos a eles\\
2. Escolher $\beta k$ produtos específicos e perguntar sua utilidade para todos os outros usuários\\
3. Classificar cada usuário

\subsubsection{Escolha das linhas}

A escolha de um número constante de linhas segue a ideia de que são necessários
alguns usuários respondendo um questionário sobre todos os produtos para que os
outros usuários sejam classificados a partir das respostas destes. Na prática, empresas
pagam pessoas para o preenchimento desses formulários.

\begin{lema}
Se selecionarmos aleatoriamente $O(k \, ln \, (k) )$ usuários, a probabilidade
de que algum tipo efetivo não tenha nenhum usuário selecionado é menor que $1\%$.
\end{lema}

\paragraph{Ideia de prova:} 
A probabilidade de nenhum usuário do tipo $i$ ser selecionado é $(1 - \frac{t_i}{m})^{ak}$,
pois a probabilidade de um usuário ser selecionado é uma variável
aleatória de Bernoulli de parâmetro $\frac{t_i}{m}$. Utilizando o limite da união e selecionando $a = O(log\, k)$, encontramos o limite proposto no enunciado.
$\\ \\ \\$
Portanto a partir da escolha das $ak$ linhas, podemos formar uma matriz $V$ de tipos efetivos.

\subsubsection{Escolha das colunas}

A ideia de escolher um número constante de colunas acontece pois todo usuário
que deseja obter uma recomendação deve responder inicialmente um pequeno 
questionário, dando alguma informação sua para que o algoritmo de recomendação
possa atual de maneira mais precisa e efetiva.

A partir da matriz $V$ calculada previamente, podemos selecionar as $\beta k$ colunas de maneira em que cada  
coluna $i$ tem probabilidade $\frac{\vert V^{(i)} \vert^2}{\Vert V \Vert^2_F}$ de ser selecionada. Com 
isso, escolheremos os produtos "mais pesados", que ajudarão a decidir se um usuário está ou não naquele tipo.

\begin{lema}
Se for descoberta a utilidade para os $\beta k$ produtos selecionados para um vetor
de x de utilidade de um usuário, pode-se aproximar $V x$ por \~v tal que 
\begin{equation}
\vert V x  - \tilde v \vert^2_2 \leq \frac{1}{\beta p}
\end{equation}
com probabilidade $1 - p$, $0 < p < 1$.
\end{lema}

\paragraph{Ideia da prova\\}

Seja a aproximação $\tilde v$ tal que $\tilde v_i = \frac {V^{(i)}}{\sqrt{(\beta k p_i)}} \frac{x_i}{\sqrt{(\beta k p_i)}}$.
 A partir disso podemos provar a aproximação de $\tilde v$ e $Vx$ aplicando a desgualdade de Markov para a variável 
aleatória $\tilde v$.


\begin{coro}
Se $\beta = \frac{10}{9p}$ e a matriz é $0.1$-separado com probabilidade $1 - p$
pode-se classificar todos os usuários pertencentes aos tipos efetivos. Logo o 
algoritmo é $\lambda$-competitivo  para utilidade.
\end{coro}

\subsection{Aproximando por amostras}

Agora será descrito um algoritmo para matriz em que não assumimos as restrições
do algoritmo anterior. Somente será necessário que a matriz $A$ tenha uma boa 
aproximação com $A_k$.

\subsubsection{Algoritmo}

1. Seleciona $r$ linhas de maneira uniforme \\
2. Selecionar $c$ colunas de onde a probabilidade de selecionar uma coluna $j$ ser selecionada
é  $q_j \geq \vert \frac{A^{(j)} \vert^2}{\Vert A \Vert^2_F}$ \\
3. Montar a matriz C com as colunas $\frac{A^(j)}{sqrt(cq_j)}$, $j = 1 .. c$. \\
4. Encontrar os k primeiros vetores singulares de C, denotados por $U_k$. \\
5. Criar a matriz à tal que 
$\tilde A_{(i)}  = \frac{A_{(i)}}{rm}$, se i foi uma linha selecionada ou $0^n$ caso contrário. \\
6. Retornar $\^A = \tilde U_k \tilde U_k^T \tilde A$


\begin{teo}
Seja $\sigma_t$ o $t-ésimo$ valor singular de A e $\rho$ é o posto de A. Utilizando
o algoritmo proposto acima, 
\begin{equation}
E(\Vert A - \tilde U_k \tilde U_k^T \tilde A \Vert^2_F) \leq \sum_{t = k + 1}^\rho \sigma_t^2 + 
(\sqrt\frac{k}{\beta c})  + \frac{k}{ar})\Vert A \Vert^2_F
\end{equation}
\end{teo}

\paragraph{Ideia de prova\\} 
Utilizando os lemas da TODO:referencia e  analizando TODO:terminar

\section{Improved Recommendation System}

\subsection{Introdução}
Ao contrário da estratégia anterior, baseada em SVD, que é computacionalmente
intensa, esta abordagem utiliza algoritmos combinatórios simples. 


\subsection{Descrição}

Nesta instância do problema, a utilidade do produto é um valor binário, ou seja,
cada produto é categorizado como bom ou ruim .

Nós chamamos bom(u) o conjunto de todos os produtos que são considerados bons
pelo usuário u.

Para analisar o algoritmo, assume-se que os usuários estão particionados em
classes de equivalências denominadas tipos, onde cada tipo representa um conjunto
de usuários com preferências similares. O sucesso do algoritmo depende da abundâcia
de tipos grandes. 

\begin{definicao}
Um tipo $T$ é um conjunto de usuários $U(T)$ e um conjunto de produtos $P(T)$ tal que
todos os produtos de $P(T)$ são bons para os usuários de $U(T)$.
\end{definicao}

\begin{definicao}
Seja $\mathcal{T}$ uma coleção de tipos, $U(\mathcal{T}) = \bigcup _{T \in \mathcal{T}} U(T)$ 
\end{definicao}

\begin{definicao}
Seja $0 < \lambda \leq 1$ e $k > 0$. Uma coleção de tipos $\mathcal{T}$ é uma $(\lambda, k)$-cobertura-de-tipos se $\vert U(\mathcal{T}) \vert \geq \lambda m$ e $\vert  U(T) \vert \geq \frac{m}{k}$ para todo tipo $T$ de $\mathcal{T}$. 
\end{definicao}


\begin{teo}
  Seja $\gamma > 0 $. Se existir uma $(\lambda, k)$-cobertura-de-tipos para algum $\lambda > 0 $ e $k > 0$, então com probabilidade pelo menos $1 - \gamma$ o algoritmo $\mathcal{A}_centr(k, \gamma)$ gera um conjunto de recomendações que satisfaz $\lambda m$ usuários com complexidade de recomendação e tempo de $O(k(m+n)log\frac{k}{\gamma})$.
\end{teo}

\addcontentsline{toc}{section}{\bibname}
\bibliographystyle{plain}
\bibliography{balancedAllocation}
\end{document}
