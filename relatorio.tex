\documentclass[a4paper,10pt]{article}
\usepackage[brazilian]{babel}
\usepackage[utf8]{inputenc}
\usepackage[T1]{fontenc}
\usepackage{amsmath}
\usepackage{amssymb}

\title{Sistemas de recomendações}
\author{Alex Grilo \\ Orientador Flávio Keidi Miyazawa\\ MC032 - Estudo Dirigido \\ \normalsize{Instituto de Computação -- Universidade Estadual de Campinas}}

\begin{document}

\maketitle

\newtheorem{definicao}{Definição}
\newtheorem{lema}{Lema}
\newtheorem{coro}{Corolário}
\newtheorem{teo}{Teorema}


\section{Introdução}

O objetivo deste relatório é descrever os assuntos estudados durante a 
disciplina de MC032 no segundo semestre de 2010. 

O principal interesse deste trabalho foi pesquisar na listeratura este
problema diferentes abordagens para resolvê-lo.

TODO:explicar estrutura do relatório

\section{Sistemas de recomendações}

Um sistema de recomendação considera quais produtos cada usuário
escolheu no passado e tenta deduzir que outros produtos um determinado
usuário pode estar interessado. 


\section{Competitive Recommendation Systems}


\subsection{Introdução}
A ideia basica da primeira abordagem é baseada nas técnicas de 
reconstrução da matriz a partir de informações parciais
da mesma. 

Nela, foram utilizadas as técnicas de reconstrução a partir
de técnicas de SVD.

\subsection{Notação}

\begin{itemize}
\item[$A$] matriz de recomendação original
\item[$A_{(i)}$] $i$-ésima linha da matriz A. No caso da matriz de recomendações, é o vetor de utilidades do $i$-ésimo usuário.
\item[$A^{(i)}$] $i$-ésima coluna a matriz A. No caso da matriz de recomendações, é a utilidade de um produto para todos os usuários.
\item[$A_{ij}$] valor da utilidade do projeto j para o usuário i
\item[$a_{ir}$] valor da utilidade do $r$-ésimo produto com maior utilidade para o usuário $i$ 
\end{itemize}

\subsection{Definições}

\begin{definicao} \label{definicao:box} Um produto j é dito bom para um usuário $i$ se $A_{ij} > a_{ir} - \delta$para o $r$ constante e $\delta$ pequeno.
\end{definicao}
\begin{definicao} \label{definicao:box}Uma recomendação é dita boa para um usuário $i$ se contém pelo menos um produto bom para aquele usuário.
\end{definicao}

\subsection{Reconstrução de matrizes e boas recomendações}

\begin{lema} \label{lema:box}
Dado que existe uma aproximação  tal que $\Vert A - $Â$ \Vert \le \epsilon \Vert A \Vert^2_F$, a probabilidade de uma recomenda\c{c}\~ao
ruim é 
\\ \\ 
$Pr [ $recomendação ruim$ ] \leq \frac{2\varepsilon}{r\delta^2}$

\end{lema}

\paragraph{Ideia da prova:}
  
A menor contribuição de uma recomendação ruim para o erro de $\Vert A - 
$Â$ \Vert \le \Vert A \Vert^2_F$ é quando os $r$ produtos com maior utilidade 
de $A_{(i)}$ possuem valor igual a $u$ e os $r$ produtos com maior utilidade
dentre os outros produtos possuem utilidade $u - \delta$.
 
Para uma recomendação ser ruim, os $r$ produtos melhor avaliados de Â$_{(i)}$
possuem utilidade $\leq x$ e os $r$ produtos seguintes possuem utilidade
$\geq x$. No caso extremo, isso acontece quando $x = u - \frac{\delta}{2}$. 

Calculando o erro a partir desses dados, assumindo que $\lambda m$ usuários
tem uma recomendação ruim e sabendo que o erro é limitado por $\epsilon \Vert
A \Vert^2_F$, obtem-se o resultado proposto no enunciado.

\subsection{Modelo de usuários}

Assumimos que existem $l$ tipos de usuários, caracterizados
pelos vetores $v^{(1)}$, ..., $v^{(l)} \in $ [$0,1$]$^n$, 
e que cada vetor é unitário e todos são "bem separados".
Intuitivamente, isso significa que os tipos
são linearmente independentes. 

\begin{definicao} \label{definicao:box} Os vetores $v^{(1)}$, ..., $v^{(l)}$
são chamados $\delta$ -separados se para cada par $(i,j)$ tal que $i \neq j$,
$v^{(i)} . v^{(j)} \leq \delta $
\end{definicao}

Chamaremos $t_j$ o número de usuários do tipo $j$. 

Assumiremos que os tipos estão ordenados pela ordem decrescente do número de usuários e que os usuários estão ordenados na matriz $A$ na ordem crescente do tipo ao qual pertencem.  

\begin{definicao} \label{definicao:box} Uma matriz de preferência A é dita $(\lambda,k)$-efetiva se 
\\ \\
$\sum^k_{i=1} t_i \geq \lambda m$
\end{definicao}

\begin{lema} \label{lema:box}
Para uma matriz $(\lambda,k)$-efetiva na forma de A 
$\\ \\$
 $\Vert A - $Â$ \Vert \le ( 1 - \lambda ) \Vert A \Vert^2_F$ 
\end{lema}
\paragraph{Ideia de prova:}

Pela construção da matriz $A$, podemos construir a matriz $B_k$ somente
com os vetores dos primeiros $k$ tipos da matriz. A matriz $B_k$ possui
aproximação 

$\Vert A - B_k\Vert^2_F \leq ( 1 - \lambda ) \Vert A \Vert^2_F$.

Portanto a matriz $A_k$, a melhor aproximação de posto $k$ da matriz A, tem uma
aproximação pelo menos igual a anterior.

\subsection{Desvio nas utilidades dos produtos}

Consideraremos que a partir da matriz $A$ proposta na subseção anterior, é
adicionado um erro, pois no mundo real, os produtos não possuem exatamente
a mesma utilidade para todos usuários do mesmo tipo.

 Modelaremos o erro adicionando à matriz $A$ uma matriz de erro $E$ tal
que $E_{ij}$ é uma variável aleatória de média 0 e variância 
$O(\frac{\epsilon^2}{m} + n )$ para $0 < \epsilon < 1$.

Chamaremos de à a matriz $ A + E $.

\subsection{Limitantes na aproximação de posto menor}

O primeiro objetivo é provar que Ã$_k$ e $A$ são próximas, assim poderemos
recriar de forma eficiente a matriz $A$ e fazer boas recomendações de 
produtos. 

\begin{lema}
 Se $\sigma_1, ..., \sigma_k$ os valores singulares de $A$, $A$ é $\delta$-separada com $\delta = O(\frac{1}{n})$ e $\frac{t_k}{t_{k+1}} \geq \beta_1 \frac{t_1}{t_k}$, onde $\beta_1$ é uma constante grande, então 
$\\ \\$
$\frac{\sigma_k}{\sigma_{k+1}} \geq \beta_2 \frac{\sigma_1}{\sigma_{k}}$
$\\ \\$
para alguma constante $\beta_2 = O (sqrt(\beta_1))$.

\end{lema}
\paragraph{Ideia da prova:} Analisando $A_\delta$ e $A_0$, pode-se verificar que $A_\delta A_\delta^T - A_0 A_0^T$ possui 2-norma no máximo $n \delta$.

Aplicando teoria padrão de perturbação para valores singulares de matrizes simétricas, pode-se concluir que os valores singulares de $A_\delta$ são perturbados somente por uma constante. Escolhendo $\beta_1$ e $\beta_2$ cuidadosamente o resultado é válido. 

$\\ \\ \\$
Utilizando o lema acima e os lemas do TODO:por bibliografia, obtemos os seguintes corolários:
  
\begin{coro} 
Como $\vert E \vert_2 = O(\epsilon)$, com probabilidade  $ 1 - o(1) $

$\vert ($Ã$_k)_{(i)} - (A_k)_{(i)}\vert_2 \geq O(\epsilon)$ \end{coro}


\begin{coro} Com probabilidade $1- o(1)$, temos 

$\Vert$ Ã$_k - A_k\Vert^2_F \geq O(\epsilon)m \geq O(\epsilon)\Vert A \Vert^2_F $
\end{coro}

\begin{coro}
E se a matriz A é $(\lambda, k)$-efetiva, segue que 

$\Vert$ Ã$_k - A_k\Vert^2_F \geq  O(\epsilon + 1 - \lambda)\Vert A \Vert^2_F $
\end{coro}

\subsection{Ã desconhecida}

Apesar da demonstração que Ã$_k$ e $A$ são próximas, um problema é que a
matriz à é desconhecida. Nós buscamos agora um algoritmo que com amostra de
$O(m+n)$ elementos, se consiga reconstruir a matriz e dar boas recomendações.

Assume-se que cada tipo efetivo contenha pelo menos $\frac{\lambda m}{100k}$ usuários.

\subsubsection{Algoritmo}

1. Escolher uniformemente $ak$ usuários e escolher a utilidade de todos produtos a eles

2. Escolher $\beta k$ produtos específicos e perguntar sua utilidade para todos os outros usuários

3. Classificar cada usuário

\subsubsection{Escolha das linhas}

A escolha de um número constante de linhas, segue a ideia de é ncessário que
usuários que respondam um questionário sobre todos os produtos para que os
outros usuários sejam classificados a partir destes. Na prática, empresas
pagam pessoas para o preenchimento desses formulários.

\begin{lema}
Se selecionarmos aleatoriamente $O(k ln (k) )$ usuários, a probabilidade
de que algum tipo efetivo não tenha nenhum usuário selecionado é menor que $1\%$.
\end{lema}

\paragraph{Ideia de prova:} 
A probabilidade de nenhum usuário do tipo $i$ ser selecionado é $(1 - \frac{t_i}{m})^{ak}$,
pois a probabilidade de um usuário ser selecionado é uma variável
aleatória de Bernoulli de parâmetro $\frac{t_i}{m}$. Utilizando o limite da união e selecionando $a = O(log k)$, encontramos o limite proposto no enunciado.


Portanto a partir da escolha das $ak$ linhas, podemos formar a matriz $V$ de tipos efetivos.

\subsubsection{Escolha das colunas}

A ideia de escolher um número constante de colunas acontece pois todos os usuários que desejam obter uma recomendação devem responder inicialmente um pequeno 
questionário, dando alguma informação sua para que seja receba recomendações precisas e efetivas.

A partir da matriz $V$ calculada previamente, podemos selecionar as $\beta k$ colunas, de uma maneira em que cada  
coluna $i$ tem probabilidade $\frac{\vert V^{(i)} \vert^2}{\Vert V \Vert^2_F}$ de ser selecionada. Com 
isso, escolheremos os produtos "mais pesados", que ajudarão a decidir se um usuário está ou não naquele tipo.

\begin{lema}
Se for descoberta a utilidade para os $\beta k$ produtos selecionados para um vetor
de x de utilidade de um usuário, pode-se aproximar $V x$ por \~v tal que 

$\vert V x  - $\~v$ \vert^2_2 \leq \frac{1}{\beta p}$

com probabilidade $1 - p$, $0 < p < 1$.
\end{lema}

\paragraph{Ideia da prova:}

Seja a aproximação \~v tal que \~v$_i = \frac {V^{(i)}}{sqrt(\beta k p_i)} \frac{x_i}{sqrt(\beta k p_i)}$.
 A partir disso podemos provar a aproximação de \~v e $Vx$ a partir da esperança e variância de \~v.


\begin{coro}
Se $\beta = \frac{10}{9p}$ e a matriz é $0.1$-separado com probabilidade $1 - p$
nós podemos classificar todos os usuários pertencentes aos tipos efetivos. Logo o 
algoritmo é $\lambda$-competitivo  para utilidade.
\end{coro}

\subsection{Aproximando por amostras}

Agora será descrito um algoritmo para matriz em que não assumimos as restrições
do algoritmo anterior. Somente será necessário que a matriz $A$ tenha uma boa 
aproximação com $A_k$.

\subsubsection{Algoritmo}

1. Seleciona $r$ linhas de maneira uniforme

2. Selecionar $c$ colunas de onde a probabilidade de selecionar uma coluna $j$ ser selecionada 
é  $q_j \geq \vert \frac{A^{(j)} \vert^2}{\Vert A \Vert^2_F}$

3. Montar a matriz C com as colunas $\frac{A^(j)}{sqrt(cq_j)}$, $j = 1 .. c$.

4. Encontrar os k primeiros vetores singulares de C, denotados por $U_k$.

5. Criar a matriz à tal que 

Ã$_{(i)}  = \frac{A_{(i)}}{rm}$, se i foi uma linha selecionada ou $0^n$ caso contrário.

6. Retornar $\^A = $~U$_k$~U$_k^T$Ã


\begin{teo}
Seja $\sigma_t$ o $t-ésimo$ valor singular de A e $\rho$ é o posto de A. Utilizando
o algoritmo proposto acima, 

$E(\Vert A - $~U$_k$~U$_k^T$Ã$ \Vert^2_F) \leq \sum_{t = k + 1}^\rho \sigma_t^2 + 
(sqrt(\frac{k}{\beta c}))  + \frac{k}{ar})\Vert A \Vert^2_F
$
\end{teo}

\paragraph{Ideia de prova:} Utilizando os lemas da TODO:referencia e 
analizando TODO:terminar

\section{Improved Recommendation System}

\subsection{Descrição}

Nesta instância do problema, a utilidade do produto é um valor binário, ou seja, cada produto é categorizado como bom ou ruim .

Nós chamamos bom(u) o conjunto de todos os produtos que são considerados bons pelo usuário u.

Para analisar o algoritmo, assume-se que os usuários estão particionados em classes de equivalências denominadas tipos, onde cada tipo representa um conjunto de usuários com preferências similares. O sucesso do algoritmo depende da abundâcia de tipos grandes. 

\begin{definicao}
Um tipo $T$ é um conjunto de usuários $U(T)$ e um conjunto de produtos $P(T)$ tal que todos os produtos de $P(T)$ são bons para os usuários de $U(T)$.
\end{definicao}

\begin{definicao}
Seja $\mathcal{T}$ uma coleção de tipos, $U(\mathcal{T}) = \bigcup _{T \in \mathcal{T}} U(T)$ 
\end{definicao}

\begin{definicao}
Seja $0 < \lambda \leq 1$ e $k > 0$. Uma coleção de tipos $\mathcal{T}$ é uma $(\lambda, k)$-cobertura-de-tipos se $\vert U(\mathcal{T}) \vert \geq \lambda m$ e $\vert  U(T) \vert \geq \frac{m}{k}$ para todo tipo $T$ de $\mathcal{T}$. 
\end{definicao}


\begin{teo}
  Seja $\gamma > 0 $. Se existir uma $(\lambda, k)$-cobertura-de-tipos para algum $\lambda > 0 $ e $k > 0$, então com probabilidade pelo menos $1 - \gamma$ o algoritmo $\mathcal{A}_centr(k, \gamma)$ gera um conjunto de recomendações que satisfaz $\lambda m$ usuários com complexidade de recomendação e tempo de $O(k(m+n)log\frac{k}{\gamma})$.
\end{teo}

\addcontentsline{toc}{section}{\bibname}
\bibliographystyle{plain}
\bibliography{balancedAllocation}
\end{document}
